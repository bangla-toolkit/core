\chapter{Conclusion}

This thesis has described the design, development, and implementation of Byakaron, a comprehensive Bangla Grammar and Spell Checker system. The objective of this project was to tackle the large gaps in the current state-of-the-art in Bangla Natural Language Processing, particularly in the realms of spell checking and grammar correction.

\section{Summary}

\subsection{Major Achievements}
The project has achieved the following:

\begin{enumerate}
    \item \textbf{Construction of the Complete Dataset}: Processed the entire Bengali Wikipedia dump, extracting more than 2 million sentences and creating a vocabulary close to 550,000 distinct words in Bangla. The bigram statistics of words with more than 5 million entries offers a strong statistical basis for language modeling.
    
    \item \textbf{NLP Tool Development}: Developed module packages for basic NLP tasks such as tokenization, text normalization, and basic stemming. These tools are designed to be extendable and could be used for other Bangla NLP projects.
    
    \item \textbf{Hybrid Spell Checker}: Designed a spell checking tool that utilizes dictionary search, edit-distance matching, phonetic (Soundex) matching, and analysis of context by N-grams. The system has a detection rate of 87.3\% and 78.5\% top-1 error correctness.
    
    \item \textbf{Grammar Checker Foundation}: Laid the foundation for rule-based grammar correction, which includes 86 rules applied to agreement of subject and verb, punctuation, and word confusion.
    
    \item \textbf{Modern Web Application}: Developed a responsive, user-friendly web application using Next.js that offers real-time spell checking with sub-second response times for typical use cases.
    
    \item \textbf{Open Source Contribution}: Released the entire project as open-source software, which helps to add to the Bangla NLP community and pave the way for further research and development.
\end{enumerate}

\subsection{Research Contributions}
The research has the following contributions to the field:

\begin{itemize}
    \item Shows the feasibility of hybrid methods in low-resource language processing
    \item Offers a processed corpus of the Bangla language for researchers to use
    \item Sets the baseline on Bangla spell checking performance
    \item Provides a useful and accessible resource for Bangla writers and learners
\end{itemize}

\section{Limitations and Future Works}

\subsection{Current Limitations}
The study has some limitations that should be acknowledged:

\begin{enumerate}
    \item \textbf{Grammar Check Completion}: No more than 10\% of grammar checking functionality has been implemented. The rule-based approach, although very precise, has limited coverage.
    
    \item \textbf{Vocabulary Coverage}: The Wikipedia corpus may not fully represent colloquial, informal, or domain-specific words in Bangla.
    
    \item \textbf{Context Window}: The current bigram model has a small context window. For longer-range dependencies, more complex models are required.
    
    \item \textbf{Processing Speed}: Very long texts (1000+ words) take multiple seconds to process, which can affect user experience in some applications.
    
    \item \textbf{Dialect Handling}: The system is designed to handle standard written Bangla and does not specifically address regional dialects.
\end{enumerate}

\subsection{Future Research Directions}
With these lessons learned and the current limitations, we propose the following future research directions:

\subsubsection{Grammar Checker Enhancement}
\begin{itemize}
    \item Full implementation of rule-based grammar checking with comprehensive coverage
    \item Investigate the use of sequence-to-sequence models (BanglaT5 \cite{bhattacharjee2022banglat5}) for grammar correction
    \item Integration with the VAIYAKARANA benchmark \cite{bhattacharyya2024vaiyakarana} for systematic assessment
\end{itemize}

\subsubsection{Model Improvements}
\begin{itemize}
    \item Use trigrams and extended n-grams to better model context sensitivity
    \item Investigate character-level embeddings for better out-of-vocabulary word handling
    \item Fine-tune BanglaBERT \cite{bhattacharjee2022banglabert} to perform contextual spell checking
\end{itemize}

\subsubsection{Dataset Expansion}
\begin{itemize}
    \item Integration of data from newspapers, social media, and literature
    \item Construction of annotated error corpora for training
    \item Develop dialect-specific materials
\end{itemize}

\subsubsection{Platform Extensions}
\begin{itemize}
    \item Create mobile applications for Android and iOS
    \item Develop browser extensions for real-time spell checking
    \item Create plugins for popular text editors such as VS Code or Microsoft Word
    \item Make public APIs for external integration
\end{itemize}

\subsubsection{Advanced Features}
\begin{itemize}
    \item Collaborative editing with spell checking
    \item Domain-specific models (technical, legal, medical writing)
    \item Style and tone recommendations beyond error correction
    \item Integration with translation and transliteration services
\end{itemize}

\section{Recommendations}
From the development experiences in the Byakaron project, the following recommendations can be made:

\subsection{For Researchers}
\begin{enumerate}
    \item \textbf{Benchmark Utilization}: Make use of standardized benchmarks, for example, VAIYAKARANA for fair comparison of grammar correction systems.
    
    \item \textbf{Data Sharing}: Publish the processed data to facilitate research on Bangla NLP.
    
    \item \textbf{Hybrid Models}: Think about integrating statistical models and neural networks instead of relying on a single approach.
\end{enumerate}

\subsection{For Developers}
\begin{enumerate}
    \item \textbf{Modular Design}: Create code that is reusable and well-documented and that can be incorporated into various applications.
    
    \item \textbf{Performance Optimization}: Apply caching and lazy loading for responsive user experiences.
    
    \item \textbf{Open Source Contribution}: Contribute to and build upon existing Bangla NLP projects rather than starting from scratch.
\end{enumerate}

\subsection{For the Community}
\begin{enumerate}
    \item \textbf{Feedback and Testing}: Users must feed back errors and edge conditions to assist in improving the system.
    
    \item \textbf{Rule Contributions}: Contributions of native speakers may include grammar rules and terms not found in Wikipedia.
    
    \item \textbf{Localization}: Encourage localization of tools to be applicable to local dialects.
\end{enumerate}

\section{Final Remarks}
Byakaron is an important milestone in the development of comprehensive Bangla language processing tools. Although spell-checking has been well-advanced, full grammar correction is still a topic for future research. The open-source character of the project will make sure that it has the capability to develop further through the contributions of the community and research extensions.

This project shows that using contemporary technologies and design principles, it is possible to develop useful NLP tools for under-resourced languages like Bangla. It is our hope that Byakaron will prove to be both a helpful aid for Bangla writers and a foundation for further research in Bangla natural language processing.

\vspace{1em}
\begin{center}
\textit{\bn{``ভাষার প্রতি ভালোবাসা থেকেই সুন্দর সাহিত্য জন্ম নেয়।''}}

(Beautiful literature is born from the love of language.)
\end{center}
