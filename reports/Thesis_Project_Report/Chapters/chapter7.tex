\chapter{Conclusion}

This thesis has discussed the design, development, and implementation of Byakaron, a comprehensive Bangla Grammar and Spell Checker system. The main aim of this project is to bridge the huge gap that is present in the state-of-the-art in Bangla Natural Language Processing, especially in the areas of spell checking and grammar checking.

\section{Summary}

\subsection{Major Achievements}
The project has achieved the following:

\begin{enumerate}
    \item \textbf{Construction of the Complete Dataset}: The entire Bengali Wikipedia dump was processed to yield more than 2 million sentences. The vocabulary is close to 550,000 distinct words in the Bangla language. The statistics of bigram words with more than 5 million entries provide strong statistical grounds for language modeling.
    
    \item \textbf{NLP Tool Development}: The module packages for the basic NLP tools were developed. The tools are extendable and can be used for other Bangla NLP projects.
    
    \item \textbf{Hybrid Spell Checker}: The spell checking tool was designed. The tool makes use of dictionary search, edit-distance match, soundex match, and analysis of context via N-grams. The system has a detection rate of 87.3\% and 78.5\% top-1 error correctness.
    
    \item \textbf{Grammar Checker Foundation}: The foundation for the rule-based grammar checker was laid. The grammar checker includes 86 rules that are used in the process of agreement.
    
    \item \textbf{Modern Web Application}: The project developed a modern web application using Next.js. The application is very user-friendly and provides real-time spell checking with sub-second response times.
    
    \item \textbf{Open Source Contribution}: The entire project was released as open-source software. The project contributes significantly to the Bangla NLP community.
\end{enumerate}

\subsection{Research Contributions}
The research contributes to the field in the following ways:

\begin{itemize}
    \item Demonstrates the viability of hybrid approaches for low-resource languages
    \item Provides a pre-processed corpus for the Bangla language for other researchers to use
    \item Establishes the baseline for spell checking performance on the Bangla language
    \item Presents a useful tool for Bangla speakers, both native and non-native writers
\end{itemize}

\section{Limitations and Future Works}

\subsection{Current Limitations}
The research also has some limitations that must be mentioned:

\begin{enumerate}
    \item \textbf{Grammar Check Completion}: Only 10\% of the grammar check functionality is implemented. Although the rule-based approach is extremely accurate, it is also limited.
    
    \item \textbf{Vocabulary Coverage}: The Wikipedia corpus might not include all the colloquial, informal, or specialized vocabulary of the Bangla language.
    
    \item \textbf{Context Window}: The current bigram model also has a narrow context window. For more complex models, longer dependency ranges are necessary.
    
    \item \textbf{Processing Speed}: For very long texts (over 1000 words), the processing time is around multiple seconds.
    
    \item \textbf{Dialect Handling}: The current approach is specifically for standard written Bangla.
\end{enumerate}

\subsection{Future Research Directions}
After identifying the lessons learned from this research, we propose the following future research directions:

\subsubsection{Grammar Checker Enhancement}
\begin{itemize}
    \item Complete implementation of the grammar checker using the rule-based approach
    \item Use sequence-to-sequence models (BanglaT5 \cite{bhattacharjee2022banglat5}) for grammar correction
    \item Integrate the VAIYAKARANA benchmark \cite{bhattacharyya2024vaiyakarana} for grammar correction
\end{itemize}

\subsubsection{Model Improvements}
\begin{itemize}
    \item Use trigrams and extended n-gram models for better context sensitivity
    \item Explore character-level embeddings for better out-of-vocabulary word handling
    \item Fine-tune the BanglaBERT model \cite{bhattacharjee2022banglabert} for contextual spell checking
\end{itemize}

\subsubsection{Dataset Expansion}
\begin{itemize}
    \item Integrate newspaper, social media, and literature data
    \item Create annotated error datasets for model training
    \item Develop dialect-specific datasets
\end{itemize}

\subsubsection{Platform Extensions}
\begin{itemize}
    \item Develop mobile applications for Android and iOS platforms
    \item Develop browser extensions for real-time spell checking
    \item Develop plugins for popular text editors such as VS Code or MS Word
    \item Develop public APIs for integration with other applications
\end{itemize}

\subsubsection{Advanced Features}
\begin{itemize}
    \item Collaborative editing with spell checking
    \item Domain-specific models for technical, legal, medical writing, etc.
    \item Style and tone suggestions
    \item Translation and transliteration integration
\end{itemize}

\section{Recommendations}
The experiences gathered during the development process of the Byakaron project can be used to provide the following recommendations for future projects:

\subsection{For Researchers}
\begin{enumerate}
    \item \textbf{Benchmark Utilization}: Utilize the available benchmarks, such as the VAIYAKARANA for grammar correction.
    
    \item \textbf{Data Sharing}: Share the processed data for further research on Bangla NLP.
    
    \item \textbf{Hybrid Models}: Consider the integration of hybrid models, such as the integration of statistical models and neural networks.
\end{enumerate}

\subsection{For Developers}
\begin{enumerate}
    \item \textbf{Modular Design}: Design the code in such a manner that it can be reused in different applications.
    
    \item \textbf{Performance Optimization}: Use caching and lazy loading for better performance.
    
    \item \textbf{Open Source Contribution}: Contribute to existing projects on Bangla NLP rather than building from scratch.
\end{enumerate}

\subsection{For the Community}
\begin{enumerate}
    \item \textbf{Feedback and Testing}: The user must provide feedback on errors and edge cases to help improve the application.
    
    \item \textbf{Rule Contributions}: Contributions from native speakers could be grammar rules and vocabulary not found on Wikipedia.
    
    \item \textbf{Localization}: Localization of the application is recommended to be applicable to the local dialect.
\end{enumerate}

\section{Final Remarks}
Byakaron is an important milestone in the development of comprehensive Bangla language processing tools. Though the spell checking has been well advanced, grammar correction is still an area to be researched in the future. The open-source nature of the project ensures that it has the potential to grow and develop further through the contributions of the community and the extensions provided through research.

This project demonstrates the potential of using modern technology and design principles to create useful NLP tools for under-resourced languages like Bangla. It is our hope that Byakaron will be useful to Bangla language writers and serve as a foundation for further research into Bangla natural language processing.

\vspace{1em}
\begin{center}
\textit{\bn{``ভাষার প্রতি ভালোবাসা থেকেই সুন্দর সাহিত্য জন্ম নেয়।''}}

(Beautiful literature is born from the love of language.)
\end{center}
