\chapter{Results and Discussion}

This chapter will discuss the results of the development of the Byakaron project. It will analyze the results compared to the expected targets. It comprises an assessment of the performance, comparison with existing tools, and a discussion of the implications and limitations of the findings.

\section{Project Completion Status}

\subsection{Overall Progress}
The status of completion of major project components till date is provided in Table \ref{tab:completion} as of the pre-defense milestone.

\begin{table}[ht]
    \centering
    \caption{Status of completion of components.}
    \begin{tabular}{l c l}
        \toprule
        \textbf{Component} & \textbf{Completion} & \textbf{Status} \\
        \toprule
        Dataset Processing & 100\% & Complete \\
        \midrule
        Database Schema & 100\% & Complete \\
        \midrule
        Core NLP Engines & 50\% & In Progress \\
        \midrule
        Spell Checker & 60\% & In Progress \\
        \midrule
        Grammar Checker & 10\% & Initial Phase \\
        \midrule
        Web Application & 70\% & In Progress \\
        \midrule
        Documentation & 80\% & Near Complete \\
        \bottomrule
    \end{tabular}
    \label{tab:completion}
\end{table}

\subsection{Dataset Results}
The data processing part of the dataset has now been fully finished with the following results:

\begin{table}[ht]
    \centering
    \caption{Final dataset statistics.}
    \begin{tabular}{l r}
        \toprule
        \textbf{Metric} & \textbf{Achieved Value} \\
        \toprule
        Source Articles (Bengali Wikipedia) & 169,736 \\
        \midrule
        Processed Sentences & 2,147,328 \\
        \midrule
        Unique Words in Dictionary & 547,892 \\
        \midrule
        Word Pairs (Bigrams) & 5,432,156 \\
        \midrule
        Grammar Rules Defined & 156 \\
        \midrule
        Database Size & 4.2 GB \\
        \bottomrule
    \end{tabular}
    \label{tab:dataset_results}
\end{table}

\section{Spell Checking Results}

\subsection{Accuracy Evaluation}
The spell checking system was tested on a test set of 1,000 sentences containing spelling mistakes. The results are shown in Table \ref{tab:spell_accuracy}.

\begin{table}[ht]
    \centering
    \caption{Spell checking accuracy measures.}
    \begin{tabular}{l c}
        \toprule
        \textbf{Metric} & \textbf{Value} \\
        \toprule
        Error Detection Rate (Recall) & 87.3\% \\
        \midrule
        Detection Precision & 91.2\% \\
        \midrule
        Correction Accuracy (Top-1) & 78.5\% \\
        \midrule
        Correction Accuracy (Top-3) & 89.7\% \\
        \midrule
        False Positive Rate & 8.8\% \\
        \midrule
        Average Response Time & 45.6 ms/sentence \\
        \bottomrule
    \end{tabular}
    \label{tab:spell_accuracy}
\end{table}

\subsubsection{Error Detection}
The system attains a recall rate of 87.3\% for identifying misspelled words, which means about 9 out of 10 spelling mistakes are identified correctly. Precision of 91.2\% shows that when the system marks a word as incorrect, it is actually correct more than 90\% of cases.

\subsubsection{Correction Accuracy}
In the case of the top recommendation offered by the system, 78.5\% of corrections are correct. This goes up to 89.7\% for the top three recommendations, meaning that the correct answer is always among the options even if it is not the first choice.

\subsection{Performance by Error Type}
The system performance for various types of errors has been shown in Table \ref{tab:error_types}.

\begin{table}[ht]
    \centering
    \caption{Spell checking performance based on error type.}
    \begin{tabular}{l c c}
        \toprule
        \textbf{Error Type} & \textbf{Detection} & \textbf{Correction} \\
        \toprule
        Single-character typos & 94.2\% & 88.3\% \\
        \midrule
        Phonetic errors & 82.1\% & 75.6\% \\
        \midrule
        Missing characters & 89.4\% & 81.2\% \\
        \midrule
        Extra characters & 91.7\% & 79.4\% \\
        \midrule
        Character substitution & 85.3\% & 72.8\% \\
        \bottomrule
    \end{tabular}
    \label{tab:error_types}
\end{table}

The system works best on simple typos and additional character errors, where the edit distance is minimal. Phonetic errors, which are errors of spelling that are phonetically similar to the correct word, prove more difficult but are nonetheless dealt with in a reasonable accuracy through the Soundex matching algorithm.

\section{Grammar Checking Results}
The grammar checking module is in its first stage of development, which is at 10\% completion, concentrating on rule-based patterns for common errors:

\begin{table}[ht]
    \centering
    \caption{Grammar rule types implemented.}
    \begin{tabular}{l c l}
        \toprule
        \textbf{Category} & \textbf{Rules} & \textbf{Status} \\
        \toprule
        Subject-Verb Agreement & 23 & Implemented \\
        \midrule
        Punctuation Errors & 18 & Implemented \\
        \midrule
        Common Word Confusions & 45 & Implemented \\
        \midrule
        Case Marker Errors & 35 & In Progress \\
        \midrule
        Verb Tense & 15 & In Progress \\
        \midrule
        Word Order & 20 & Planned \\
        \bottomrule
    \end{tabular}
    \label{tab:grammar_rules}
\end{table}

\section{Web Application Results}
The Byakaron web application has a functional interface for spell checking:

\subsection{User Interface Features}
\begin{itemize}
    \item Supports up to 10,000 characters in the text input area
    \item Spell checking with highlighting of errors on-the-fly
    \item Click-to-correct feature for accepting suggestions
    \item Mobile and desktop responsive design
\end{itemize}

\subsection{Response Time Analysis}
The response times of the system in various situations are shown in Table \ref{tab:response_times}.

\begin{table}[ht]
    \centering
    \caption{Analysis of response times for web applications.}
    \begin{tabular}{l c}
        \toprule
        \textbf{Scenario} & \textbf{Avg. Response Time} \\
        \toprule
        Single word check & 12 ms \\
        \midrule
        Short sentence (5-10 words) & 45 ms \\
        \midrule
        Medium paragraph (50 words) & 180 ms \\
        \midrule
        Long text (200 words) & 620 ms \\
        \midrule
        Maximum input (1000 words) & 2.8 s \\
        \bottomrule
    \end{tabular}
    \label{tab:response_times}
\end{table}

The response time takes less than 1 second for general usage (under 200 words), ensuring an optimal user experience.

\section{Comparison with Existing Tools}

\subsection{Feature Comparison}
Comparison between Byakaron and the existing solutions for spell checking in the Bangla language is presented in Table \ref{tab:comparison}.

\begin{table}[ht]
    \centering
    \caption{Comparison with existing Bangla spell checking tools.}
    \resizebox{\textwidth}{!}{%
    \begin{tabular}{l c c c c}
        \toprule
        \textbf{Feature} & \textbf{Byakaron} & \textbf{OpenBangla} & \textbf{Avro} & \textbf{BSpell} \\
        \toprule
        Web-based & \checkmark & $\times$ & $\times$ & $\times$ \\
        \midrule
        Context-sensitive & \checkmark & $\times$ & $\times$ & \checkmark \\
        \midrule
        N-gram model & \checkmark & $\times$ & $\times$ & $\times$ \\
        \midrule
        Open source & \checkmark & \checkmark & $\times$ & \checkmark \\
        \midrule
        Grammar support & Partial & $\times$ & $\times$ & $\times$ \\
        \midrule
        API available & \checkmark & $\times$ & $\times$ & $\times$ \\
        \bottomrule
    \end{tabular}%
    }
    \label{tab:comparison}
\end{table}

\subsection{Accuracy Comparison}
On the basis of existing benchmarks as well as figures reported in literature:

\begin{itemize}
    \item Byakaron has comparable detection rates to BSpell \cite{hasan2023bspell} (87.3\% vs 89.1\%)
    \item Context with n-grams offers a superior suggestion process compared to those that use a dictionary
    \item Hybrid approach performs better than purely edit-distance methods \cite{khan2014ngram}
\end{itemize}

\section{Discussion}

\subsection{Achievement of Objectives}
Assessment with respect to the initial set of objectives:

\begin{enumerate}
    \item \textbf{Dataset Preparation}: \textit{Fully achieved}. A complete dataset of 2+ million sentences and 500K+ unique words has been processed from Wikipedia dumps.
    
    \item \textbf{NLP Tool Development}: \textit{Substantially achieved}. Tokenization, text cleaning, and basic stemming modules are functional.
    
    \item \textbf{Spell Checker Implementation}: \textit{Partially achieved (60\%)}. The core spell checking functionality works well, but optimization and edge case handling continue.
    
    \item \textbf{Grammar Checker Foundation}: \textit{Initial progress (10\%)}. Rule definitions are in place, but implementation is limited.
    
    \item \textbf{Web Application}: \textit{Mostly achieved}. The application is functional and usable, with some UI improvements pending.
\end{enumerate}

\subsection{Unexpected Findings}
Some unexpected discoveries during the stages of development include:

\begin{itemize}
    \item \textbf{Vocabulary Coverage}: Although the Wikipedia corpus is very large, it underrepresents colloquial and informal Bangla words in the dictionary.
    
    \item \textbf{Phonetic Challenges}: The similarity in most Bangla characters results in higher false positive rates than were initially expected.
    
    \item \textbf{Performance Trade-offs}: Larger n-gram contexts lead to increased accuracy but impact response time.
\end{itemize}

\subsection{Implications}
The project shows that:

\begin{enumerate}
    \item Methods combining both statistical techniques and rules are applicable for Bangla NLP
    \item Wikipedia contains enough data to create functional spell checking systems
    \item Modern web technologies support responsive NLP systems without complex infrastructure
\end{enumerate}

\section{Budget Utilization}
The budget breakdown planned for the project is shown in Table \ref{tab:budget}.

\begin{table}[ht]
    \centering
    \caption{Project budget breakdown.}
    \begin{tabular}{l r}
        \toprule
        \textbf{Category} & \textbf{Amount (BDT)} \\
        \toprule
        Developer Resources & 1,50,000 \\
        \midrule
        Computing (GPU/Cloud) & 80,000 \\
        \midrule
        Hosting \& Infrastructure & 40,000 \\
        \midrule
        Mobile App Development & 40,000 \\
        \midrule
        Marketing \& Outreach & 30,000 \\
        \midrule
        Miscellaneous & 20,000 \\
        \midrule
        \textbf{Total} & \textbf{3,60,000} \\
        \bottomrule
    \end{tabular}
    \label{tab:budget}
\end{table}

\section{Limitations}
The existing system has the following disadvantages:

\begin{enumerate}
    \item \textbf{Grammar Checking}: Only 10\% complete, not very useful for comprehensive language correction.
    
    \item \textbf{Vocabulary Bias}: The Wikipedia corpus may not cover all fields and types of writing styles.
    
    \item \textbf{Real-Word Errors}: Context-dependent errors in which the misspelled word is itself a valid word remain challenging.
    
    \item \textbf{Processing Speed}: The longer the texts are, the longer it takes to process them.
    
    \item \textbf{No Dialect Support}: Regional dialects of Bangla are not specifically addressed.
\end{enumerate}

Such limitations offer future directions for growth and study.
