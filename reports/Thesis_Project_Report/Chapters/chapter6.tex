\chapter{Results and Discussion}

This chapter presents the results of the Byakaron project development, analyzing the outcomes against the stated objectives. It includes an evaluation of the system's performance, comparison with existing tools, and a discussion of the implications and limitations of the findings.

\section{Project Completion Status}

\subsection{Overall Progress}
Table \ref{tab:completion} summarizes the completion status of major project components as of the pre-defense milestone.

\begin{table}[ht]
    \centering
    \caption{Project component completion status.}
    \begin{tabular}{l c l}
        \toprule
        \textbf{Component} & \textbf{Completion} & \textbf{Status} \\
        \toprule
        Dataset Processing & 100\% & Complete \\
        \midrule
        Database Schema & 100\% & Complete \\
        \midrule
        Core NLP Engines & 50\% & In Progress \\
        \midrule
        Spell Checker & 60\% & In Progress \\
        \midrule
        Grammar Checker & 10\% & Initial Phase \\
        \midrule
        Web Application & 70\% & In Progress \\
        \midrule
        Documentation & 80\% & Near Complete \\
        \bottomrule
    \end{tabular}
    \label{tab:completion}
\end{table}

\subsection{Dataset Results}
The dataset processing component has been fully completed with the following results:

\begin{table}[ht]
    \centering
    \caption{Final dataset statistics.}
    \begin{tabular}{l r}
        \toprule
        \textbf{Metric} & \textbf{Achieved Value} \\
        \toprule
        Source Articles (Bengali Wikipedia) & 169,736 \\
        \midrule
        Processed Sentences & 2,147,328 \\
        \midrule
        Unique Words in Dictionary & 547,892 \\
        \midrule
        Word Pairs (Bigrams) & 5,432,156 \\
        \midrule
        Grammar Rules Defined & 156 \\
        \midrule
        Database Size & 4.2 GB \\
        \bottomrule
    \end{tabular}
    \label{tab:dataset_results}
\end{table}

\section{Spell Checking Results}

\subsection{Accuracy Evaluation}
The spell checking system was evaluated on a test set of 1,000 sentences containing known spelling errors. Results are presented in Table \ref{tab:spell_accuracy}.

\begin{table}[ht]
    \centering
    \caption{Spell checking accuracy metrics.}
    \begin{tabular}{l c}
        \toprule
        \textbf{Metric} & \textbf{Value} \\
        \toprule
        Error Detection Rate (Recall) & 87.3\% \\
        \midrule
        Detection Precision & 91.2\% \\
        \midrule
        Correction Accuracy (Top-1) & 78.5\% \\
        \midrule
        Correction Accuracy (Top-3) & 89.7\% \\
        \midrule
        False Positive Rate & 8.8\% \\
        \midrule
        Average Response Time & 45.6 ms/sentence \\
        \bottomrule
    \end{tabular}
    \label{tab:spell_accuracy}
\end{table}

\subsubsection{Error Detection}
The system achieves 87.3\% recall in detecting misspelled words, meaning that approximately 9 out of 10 spelling errors are correctly identified. The precision of 91.2\% indicates that when the system flags a word as incorrect, it is correct in over 90\% of cases.

\subsubsection{Correction Accuracy}
When considering the top suggestion provided by the system, 78.5\% of corrections are accurate. This increases to 89.7\% when considering the top three suggestions, indicating that the correct word is usually among the suggestions even if not the first choice.

\subsection{Performance by Error Type}
Table \ref{tab:error_types} shows system performance across different error categories.

\begin{table}[ht]
    \centering
    \caption{Spell checking performance by error type.}
    \begin{tabular}{l c c}
        \toprule
        \textbf{Error Type} & \textbf{Detection} & \textbf{Correction} \\
        \toprule
        Typos (single character) & 94.2\% & 88.3\% \\
        \midrule
        Phonetic errors & 82.1\% & 75.6\% \\
        \midrule
        Missing characters & 89.4\% & 81.2\% \\
        \midrule
        Extra characters & 91.7\% & 79.4\% \\
        \midrule
        Character substitution & 85.3\% & 72.8\% \\
        \bottomrule
    \end{tabular}
    \label{tab:error_types}
\end{table}

The system performs best on simple typos and extra character errors, where the edit distance is minimal. Phonetic errors, where the misspelling sounds similar to the correct word, prove more challenging but are still handled with reasonable accuracy using the Soundex matching algorithm.

\section{Grammar Checking Results}
The grammar checking component is in its initial phase (10\% completion), focusing on rule-based patterns for common errors:

\begin{table}[ht]
    \centering
    \caption{Implemented grammar rule categories.}
    \begin{tabular}{l c l}
        \toprule
        \textbf{Category} & \textbf{Rules} & \textbf{Status} \\
        \toprule
        Subject-Verb Agreement & 23 & Implemented \\
        \midrule
        Punctuation Errors & 18 & Implemented \\
        \midrule
        Common Word Confusions & 45 & Implemented \\
        \midrule
        Case Marker Errors & 35 & In Progress \\
        \midrule
        Verb Tense & 15 & In Progress \\
        \midrule
        Word Order & 20 & Planned \\
        \bottomrule
    \end{tabular}
    \label{tab:grammar_rules}
\end{table}

\section{Web Application Results}
The Byakaron web application provides a functional interface for spell checking:

\subsection{User Interface Features}
\begin{itemize}
    \item Large text input area supporting up to 10,000 characters
    \item Real-time spell checking with visual error highlighting
    \item Click-to-correct functionality for accepting suggestions
    \item Responsive design for mobile and desktop use
\end{itemize}

\subsection{Response Time Analysis}
Table \ref{tab:response_times} shows the system response times under different conditions.

\begin{table}[ht]
    \centering
    \caption{Web application response time analysis.}
    \begin{tabular}{l c}
        \toprule
        \textbf{Scenario} & \textbf{Avg. Response Time} \\
        \toprule
        Single word check & 12 ms \\
        \midrule
        Short sentence (5-10 words) & 45 ms \\
        \midrule
        Medium paragraph (50 words) & 180 ms \\
        \midrule
        Long text (200 words) & 620 ms \\
        \midrule
        Maximum input (1000 words) & 2.8 s \\
        \bottomrule
    \end{tabular}
    \label{tab:response_times}
\end{table}

Response times remain under 1 second for typical use cases (under 200 words), providing a satisfactory user experience.

\section{Comparison with Existing Tools}

\subsection{Feature Comparison}
Table \ref{tab:comparison} compares Byakaron with existing Bangla spell checking solutions.

\begin{table}[ht]
    \centering
    \caption{Comparison with existing Bangla spell checking tools.}
    \begin{tabular}{l c c c c}
        \toprule
        \textbf{Feature} & \textbf{Byakaron} & \textbf{OpenBangla} & \textbf{Avro} & \textbf{BSpell} \\
        \toprule
        Web-based & \checkmark & $\times$ & $\times$ & $\times$ \\
        \midrule
        Context-sensitive & \checkmark & $\times$ & $\times$ & \checkmark \\
        \midrule
        N-gram model & \checkmark & $\times$ & $\times$ & $\times$ \\
        \midrule
        Open source & \checkmark & \checkmark & $\times$ & \checkmark \\
        \midrule
        Grammar support & Partial & $\times$ & $\times$ & $\times$ \\
        \midrule
        API available & \checkmark & $\times$ & $\times$ & $\times$ \\
        \bottomrule
    \end{tabular}
    \label{tab:comparison}
\end{table}

\subsection{Accuracy Comparison}
Based on available benchmarks and reported figures from literature:

\begin{itemize}
    \item Byakaron achieves comparable detection rates to BSpell \cite{hasan2023bspell} (87.3\% vs 89.1\%)
    \item N-gram context provides better suggestions than dictionary-only approaches
    \item The hybrid approach outperforms pure edit-distance methods \cite{khan2014ngram}
\end{itemize}

\section{Discussion}

\subsection{Achievement of Objectives}
Evaluating against the original objectives:

\begin{enumerate}
    \item \textbf{Dataset Preparation}: \textit{Fully achieved}. A comprehensive dataset of 2+ million sentences and 500K+ unique words has been processed from Wikipedia dumps.
    
    \item \textbf{NLP Tool Development}: \textit{Substantially achieved}. Tokenization, text cleaning, and basic stemming modules are functional.
    
    \item \textbf{Spell Checker Implementation}: \textit{Partially achieved (60\%)}. The core spell checking functionality works with good accuracy, but optimization and edge case handling continue.
    
    \item \textbf{Grammar Checker Foundation}: \textit{Initial progress (10\%)}. Rule definitions are in place, but implementation is limited.
    
    \item \textbf{Web Application}: \textit{Substantially achieved}. The application is functional and usable, with some UI improvements pending.
\end{enumerate}

\subsection{Unexpected Findings}
During development, several unexpected findings emerged:

\begin{itemize}
    \item \textbf{Vocabulary Coverage}: The Wikipedia corpus, while extensive, underrepresents colloquial and informal Bangla vocabulary.
    
    \item \textbf{Phonetic Challenges}: The similarity of many Bangla characters leads to higher false positive rates than initially expected.
    
    \item \textbf{Performance Trade-offs}: Increasing n-gram context size improves accuracy but significantly impacts response time.
\end{itemize}

\subsection{Implications}
The project demonstrates that:

\begin{enumerate}
    \item Hybrid approaches combining statistical and rule-based methods are viable for Bangla NLP
    \item Wikipedia provides sufficient data for building functional spell checking systems
    \item Modern web technologies enable responsive NLP applications without complex infrastructure
\end{enumerate}

\section{Budget Utilization}
Table \ref{tab:budget} shows the planned budget allocation for the project.

\begin{table}[ht]
    \centering
    \caption{Project budget allocation.}
    \begin{tabular}{l r}
        \toprule
        \textbf{Category} & \textbf{Amount (BDT)} \\
        \toprule
        Developer Resources & 1,50,000 \\
        \midrule
        Computing (GPU/Cloud) & 80,000 \\
        \midrule
        Hosting \& Infrastructure & 40,000 \\
        \midrule
        Mobile App Development & 40,000 \\
        \midrule
        Marketing \& Outreach & 30,000 \\
        \midrule
        Miscellaneous & 20,000 \\
        \midrule
        \textbf{Total} & \textbf{3,60,000} \\
        \bottomrule
    \end{tabular}
    \label{tab:budget}
\end{table}

\section{Limitations}
The current implementation has the following limitations:

\begin{enumerate}
    \item \textbf{Grammar Checking}: Only 10\% complete, limiting utility for comprehensive language correction.
    
    \item \textbf{Vocabulary Bias}: Wikipedia corpus may not represent all domains and writing styles.
    
    \item \textbf{Real-Word Errors}: Context-dependent errors where the misspelled word is itself a valid word remain challenging.
    
    \item \textbf{Processing Speed}: Very long texts require several seconds to process.
    
    \item \textbf{No Dialect Support}: Regional variations of Bangla are not specifically addressed.
\end{enumerate}

These limitations represent opportunities for future development and research.
