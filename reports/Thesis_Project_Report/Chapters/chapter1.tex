\chapter{Introduction}

\section{Background}
Bangla, the language of the Bengal region and the native tongue of more than 300 million people worldwide, ranks as the fifth most spoken language in the world. However, despite its universality in usage, Bangla is still vastly underrepresented in terms of comprehensive Natural Language Processing (NLP) tools, especially when related to spell-checking and grammar correction \cite{bhattacharyya2024vaiyakarana}. The digital transformation has raised the need for robust language processing tools to ensure error-free written communication in Bangla.

The major hurdles in the creation of Bangla NLP systems arise from the language's morphological complexity, lexical breadth, and inherent difficulties related to typing on English keyboards. Bangla has many compound words, similarly pronounced letter characters, and grammatical rules depending on context, such that automatic error detection and correction are particularly difficult \cite{hasan2023bspell}. Furthermore, the absence of comprehensive, high-quality data available for statistical and machine learning models has stifled further development in this area.

Previous attempts to create a Bangla spell checker were based mainly on either rule-based methods or simple edit distance algorithms \cite{khan2014ngram}. Although these methods offer some degree of correction ability, they fail in dealing with context-sensitive errors and struggle with the inflectional nature of Bangla vocabulary. Current breakthroughs in deep learning and transformer models are ushering in new possibilities for more accurate and context-aware language correction systems \cite{devlin2019bert, raffel2020t5}.

\section{Problem Statement}
Bangla is one of the top-ten languages in terms of the number of speakers around the globe. Despite its importance, there is a lack of comprehensive free spell checking and grammar correction tools for the language in comparison to those available for English.

The important issues that emerged are:
\begin{enumerate}
    \item \textbf{Inadequate Spell Checking Resources}: Available Bangla spell checker algorithms either lack some words or function with reference to less detailed dictionaries without presenting context-dependent proposals, resulting in poor correction accuracy \cite{pal2021ngram}.
    
    \item \textbf{Lack of Grammar Correction Tools}: While many spell checking packages exist, grammar correction tools that are comprehensive in nature for the Bangla language are virtually non-existent in the open-source community \cite{miah2023t5}.
    
    \item \textbf{Lack of Quality Datasets}: The design of modern NLP systems requires large, well-structured datasets. The lack of properly processed Bangla corpora has been a major hindrance to research and development \cite{wikidumps2024}.
    
    \item \textbf{Fragmented Tool Ecosystem}: The current set of Bangla NLP tools is fragmented, with incompatibilities and lack of standardization, which could cause difficulties for developers and researchers to build on previous research \cite{sarker2021bnlp}.
\end{enumerate}

All these make the need to implement an effective and integrated solution that spans both spell checking and grammar correction, and contributes accessible reusable datasets and tools within the Bangla NLP community.

\section{Purpose of the Project}
The goal of the thesis project is to develop a program called \textbf{Byakaron}, a comprehensive Bangla Grammar and Spell Checker software. The proposed work will aim to contribute to the Bangla NLP community as follows:

\subsection{Objectives}
\begin{enumerate}
    \item \textbf{Dataset Preparation}: Prepare properly processed and structured Bangla language dataset from the Wikipedia dumps, ranging from sentence-level data, word frequencies, and n-gram statistics.
    
    \item \textbf{NLP Tool Development}: Develop an inventory of reusable NLP tools for Bangla Natural Language Processing, including tokenization, normalization, and text cleaning utilities.
    
    \item \textbf{Spell Checker Implementation}: Develop a hybrid spell checker that integrates statistical N-grams with rule-based error correction to accomplish high accuracy for error detection and suggestion.
    
    \item \textbf{Grammar Checker Foundation}: Establish the foundation for a grammar correction system that can be extended in future endeavors.
    
    \item \textbf{Web Application}: Build an accessible and user-friendly web application that enables customers to verify and edit their Bangla text in real time.
\end{enumerate}

\subsection{Scope and Limitations}
This project aims at:
\begin{enumerate}[label=\Roman*.]
    \item Working with Bengali Wikipedia dumps as the prime dataset
    \item Spell checking with around 60\% completion status
    \item Development of grammar checking infrastructure with around 10\% completion status
    \item Creating a web interface using modern technology stacks (Next.js, TypeScript)
\end{enumerate}

\begin{itemize}
\item[\Large $\rightarrow$] The present coverage fails to include:
\end{itemize}

\begin{enumerate}[label=\Roman*.]
    \item iOS app development
    \item Facilities for collaborative editing in real-time
    \item Regional Bengali dialects support
    \item Compatibility with external writing tools
\end{enumerate}

\section{Significance of the Study}

\begin{itemize}
\item[\Large $\rightarrow$] This work is significant to the Bangla NLP community for a number of reasons:
\end{itemize}

\begin{itemize}
\item[\Large $\triangleright$] \textbf{Academic Contribution}: By presenting a well-documented hybrid solution that combines N-gram statistical models and rule-based correction, this work serves as a resource for further research in low-resource languages.

\item[\Large $\triangleright$] \textbf{Data Contribution}: The processed Wikipedia dataset, including word pairs and frequency statistics, will be accessible to the research community, hence enabling further development in Bangla NLP.

\item[\Large $\triangleright$] \textbf{Practical Application}: The web application designed within the scope of this research is of direct application to writers, students, and professionals who communicate in Bangla.

\item[\Large $\triangleright$] \textbf{Tool Standardization}: The modular nature of the project, consisting of independent packages for NLP tasks, database processing, and working with datasets, encourages code reusability and standardization in the Bangla NLP community.
\end{itemize}

\section{Organization of the Report}
This thesis report has the following organization:
\begin{itemize}
\item[\Large $\diamond$] \textbf{Chapter 1 (Introduction)}:  This chapter will include background information, the problem, objectives, and the significance of the study.

\item[\Large $\diamond$] \textbf{Chapter 2 (Literature Review)}:   This chapter will include the existing literature and resources available on the topic of Bangla spell checking and grammar correction, along with the gaps in the existing techniques.

\item[\Large $\diamond$] \textbf{Chapter 3 (Methodology)}:  This chapter will include the research design, the system architecture, the data model, and the algorithms used in the project.

\item[\Large $\diamond$] \textbf{Chapter 4 (Data Presentation and Analysis)}: This chapter will include the implementation details, the technology stack, and the data processing techniques used in the project.

\item[\Large $\diamond$] \textbf{Chapter 5 (Engineering Considerations)}: This chapter will include the various factors such as society, the environment, and the ethics involved in the project.

\item[\Large $\diamond$] \textbf{Chapter 6 (Results and Discussion)}: This chapter involves the presentation of the project outcomes, testing results, and analysis of the developed system.

\item[\Large $\diamond$] \textbf{Chapter 7 (Conclusion)}: This chapter will include the achievements, the limitations, and the recommendations for future work.
\end{itemize}
