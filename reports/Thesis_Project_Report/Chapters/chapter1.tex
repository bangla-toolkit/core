\chapter{Introduction}

\section{Background}
The Bengali (Bangla) language, spoken by over 300 million people worldwide, ranks as the fifth most spoken language globally. Despite its widespread use, Bangla remains significantly underserved in terms of comprehensive Natural Language Processing (NLP) tools, particularly in the domains of spell checking and grammar correction \cite{bhattacharyya2024vaiyakarana}. The digital transformation has increased the necessity for robust language processing tools that can ensure accurate and error-free written communication in Bangla.

The primary challenges in developing Bangla NLP tools stem from the language's morphological complexity, extensive vocabulary, and the inherent difficulties in typing using English keyboards. Bangla contains numerous compound letters, similarly pronounced characters, and context-dependent grammatical rules that make automatic error detection and correction particularly challenging \cite{hasan2023bspell}. Furthermore, the lack of comprehensive, high-quality datasets for training statistical and machine learning models has hindered progress in this domain.

Prior attempts at developing Bangla spell checkers have primarily relied on either rule-based approaches or basic edit distance algorithms \cite{khan2014ngram}. While these methods provide some level of correction capability, they often fail to handle context-sensitive errors and struggle with the inflectional nature of Bangla vocabulary. Recent advances in deep learning and transformer-based models have opened new possibilities for more accurate and context-aware language correction systems \cite{devlin2019bert, raffel2020t5}.

\section{Problem Statement}
The current landscape of Bangla language processing tools presents several critical gaps that this project aims to address. Despite being one of the most spoken languages in the world, Bangla lacks comprehensive and freely available spell checking and grammar correction systems that can match the accuracy and reliability of tools available for languages like English.

The key problems identified are:
\begin{enumerate}
    \item \textbf{Inadequate Spell Checking Resources}: Existing Bangla spell checkers either rely on limited dictionaries or fail to provide context-sensitive suggestions, resulting in poor correction accuracy \cite{pal2021ngram}.
    
    \item \textbf{Absence of Grammar Correction Tools}: While some spell checking solutions exist, comprehensive grammar correction tools for Bangla are virtually non-existent in the open-source domain \cite{miah2023t5}.
    
    \item \textbf{Lack of Quality Datasets}: The development of modern NLP systems requires large, well-structured datasets. The absence of properly processed Bangla corpora has been a significant barrier to research and development \cite{wikidumps2024}.
    
    \item \textbf{Fragmented Tool Ecosystem}: The existing Bangla NLP tools are scattered, incompatible, and lack standardization, making it difficult for developers and researchers to build upon previous work \cite{sarker2021bnlp}.
\end{enumerate}

These challenges motivate the development of a comprehensive, integrated solution that addresses both spell checking and grammar correction while contributing reusable datasets and tools to the Bangla NLP community.

\section{Purpose of the Project}
The primary objective of this thesis project is to design and develop \textbf{Byakaron}, a comprehensive Bangla Grammar and Spell Checker system. The project aims to contribute to the Bangla NLP ecosystem by providing:

\subsection{Objectives}
\begin{enumerate}
    \item \textbf{Dataset Preparation}: Create a properly processed and structured Bangla language dataset extracted from Wikipedia dumps, including sentence-level data, word frequencies, and n-gram (word pair) statistics.
    
    \item \textbf{NLP Tool Development}: Develop a suite of reusable NLP tools for Bangla language processing, including tokenization, normalization, and text cleaning utilities.
    
    \item \textbf{Spell Checker Implementation}: Build a hybrid spell checking system that combines statistical N-gram models with rule-based corrections to achieve high accuracy in error detection and suggestion.
    
    \item \textbf{Grammar Checker Foundation}: Establish the foundation for a rule-based grammar correction system that can be extended in future work.
    
    \item \textbf{Web Application}: Develop an accessible, user-friendly web application that allows users to check and correct their Bangla text in real-time.
\end{enumerate}

\subsection{Scope and Limitations}
This project focuses on:
\begin{itemize}
    \item Processing Bengali Wikipedia dumps as the primary data source
    \item Implementing spell checking with approximately 60\% completion status
    \item Establishing grammar checking infrastructure with approximately 10\% completion status
    \item Building a web-based interface using modern technologies (Next.js, TypeScript)
\end{itemize}

The current scope does not include:
\begin{itemize}
    \item Mobile application development
    \item Real-time collaborative editing features
    \item Support for regional Bengali dialects
    \item Integration with third-party writing platforms
\end{itemize}

\section{Significance of the Study}
This project contributes to the Bangla NLP ecosystem in several significant ways:

\textbf{Academic Contribution}: By providing a well-documented hybrid approach that combines N-gram statistical models with rule-based corrections, this work serves as a reference for future research in low-resource language processing.

\textbf{Dataset Contribution}: The processed Wikipedia dataset, including word pairs and frequency statistics, will be made available to the research community, enabling further development in Bangla NLP.

\textbf{Practical Application}: The web application developed as part of this project provides immediate practical value to writers, students, and professionals who communicate in Bangla.

\textbf{Tool Standardization}: The modular architecture of the project, with separate packages for core NLP functionality, database operations, and dataset processing, promotes code reuse and standardization in the Bangla NLP community.

\section{Organization of the Report}
This thesis report is organized as follows:

\textbf{Chapter 1 (Introduction)}: Provides the background, problem statement, objectives, and significance of the study.

\textbf{Chapter 2 (Literature Review)}: Reviews existing research and tools related to Bangla spell checking and grammar correction, identifying gaps in current approaches.

\textbf{Chapter 3 (Methodology)}: Describes the research design, system architecture, data model, and algorithmic approaches used in the project.

\textbf{Chapter 4 (Data Presentation and Analysis)}: Presents the implementation details, technology stack, and data processing pipelines.

\textbf{Chapter 5 (Engineering Considerations)}: Discusses the societal, environmental, and ethical considerations of the project.

\textbf{Chapter 6 (Results and Discussion)}: Presents the project outcomes, testing results, and analysis of the developed system.

\textbf{Chapter 7 (Conclusion)}: Summarizes the achievements, discusses limitations, and proposes future research directions.
