\chapter{Engineering Considerations}

Engineering aspects involve a number of key areas that engineers must consider. This chapter will discuss the various aspects of the Byakaron project from a societal, environmental, and ethical point of view to ensure that the development process is in line with good engineering practices.

\section{Societal Impacts of Engineering Solutions}
The engineer's role in relation to society involves understanding the societal consequences of engineering solutions such as safety and accessibility, and for the well-being of society.

\subsection{Language Preservation and Digital Inclusion}
Byakaron is involved in the preservation and promotion of the Bangla language in the digital age. Key impacts on society include:

\begin{itemize}
    \item \textbf{Cultural Preservation}: By developing tools that facilitate accurate writing in the language, this project helps in the preservation of linguistic heritage while promoting the use of the Bangla language in digital communication.
    
    \item \textbf{Digital Literacy}: It also helps in the creation of professional writing in the language through the incorporation of features such as spell checking and grammar correction, thus improving the level of digital literacy.
    
    \item \textbf{Educational Support}: Learners and students can utilize this tool to improve their writing skills, obtaining immediate feedback on errors and corrections.
\end{itemize}

\subsection{Accessibility}
Accessibility is a design principle that the project adopts:

\begin{enumerate}
    \item \textbf{Web-Based Access}: The application is browser-based; therefore, there is no requirement for software installation, facilitating access by any computer with internet connectivity.
    
    \item \textbf{Responsive Design}: The interface is designed to be responsive to varying screen sizes, supporting desktop computers, tablets, and mobile phones.
    
    \item \textbf{Simple Interface}: The interface is designed to be user-friendly, requiring only little technical knowledge to operate.
    
    \item \textbf{Free Access}: The application is free of charge, and this means that economic constraints will not be a barrier to access.
\end{enumerate}

\subsection{Community}
The fact that it's an open-source project enables the Bangla-speaking community to:

\begin{itemize}
    \item Contribute improvements and bug fixes
    \item Expand on the datasets and models
    \item Customize the tools to suit their organizational requirements
    \item Use the code base for learning by developers
\end{itemize}

\section{Environment and Sustainability Issues}
Issues of the environment and sustainability include evaluating and addressing the environmental implications of engineering projects while fostering sustainable practices.

\subsection{Computational Resource Optimization}
This project adopts various methods to ensure that its effects on the environment are minimal:

\begin{itemize}
    \item \textbf{Efficient Runtime Environment}: Bun offers 4x faster start-up times than Node.js, cutting down on energy usage during development and deployment.
    
    \item \textbf{Database Optimization}: Indexed queries and connection pooling help reduce database server load and energy consumption.
    
    \item \textbf{Caching Strategies}: The results of queries are cached to prevent redundant computations, thus reducing CPU usage.
    
    \item \textbf{Lazy Loading}: Resources are loaded by the application only when they are needed, rather than upfront, thus reducing the initial transfer of data.
\end{itemize}

\subsection{Cloud Resource Management}
For production deployment, the project adopts sustainable cloud practices in the following way:

\begin{enumerate}
    \item \textbf{Right-Sizing}: The server resources are right-sized based on demand rather than over-provisioned.
    
    \item \textbf{Auto Scaling}: Dynamic scaling helps to ensure that resources are activated only when they are needed.
    
    \item \textbf{Region Selection}: Preference for data centers powered by renewable energy where available.
\end{enumerate}

\subsection{Data Center Efficiency}
The technology choices promote energy-efficient use:

\begin{table}[ht]
    \centering
    \caption{Energy efficiency criteria for technology selection.}
    \begin{tabular}{l l}
        \toprule
        \textbf{Technology} & \textbf{Efficiency Benefit} \\
        \toprule
        Bun Runtime & Faster execution, lower CPU time \\
        \midrule
        PostgreSQL & Efficient query optimizer, minimal I/O \\
        \midrule
        Next.js SSR & Reduces client-side computation \\
        \midrule
        Edge Caching & Origin server requests decreased \\
        \bottomrule
    \end{tabular}
    \label{tab:efficiency}
\end{table}

\subsection{Long-Term Sustainability}
The project achieves long-term sustainability through:

\begin{itemize}
    \item \textbf{Open Source Model}: This model relies upon community support to maintain longevity without requiring constant commercial support.
    
    \item \textbf{Modular Architecture}: The components can be upgraded separately, thereby extending the life of the system.
    
    \item \textbf{Documentation}: Well-documented code makes it easier for other developers to keep and develop the project.
\end{itemize}

\section{Ethical Implications}
Engineering ethics focuses on the ethical and professional obligations of engineers. The ethical concerns in this profession include the need to promote safety, honesty, and integrity in engineering practices.

\subsection{Data Privacy and User Protection}
The project follows rigorous guidelines on data privacy:

\begin{enumerate}
    \item \textbf{No Collection of Personal Data}: The application doesn't need user accounts or collect personal data.
    
    \item \textbf{Transient Processing}: The user's text is processed in memory and not stored in databases.
    
    \item \textbf{No Tracking}: The application does not make use of analytics or tracking tools that might affect user privacy.
    
    \item \textbf{Secure Communication}: Data transmitted is encrypted using HTTPS.
\end{enumerate}

\subsection{Transparency and Honesty}
The project ensures transparency in the following ways:

\begin{itemize}
    \item \textbf{Open Source Code}: The source code is open for public review and audit.
    
    \item \textbf{Clear Limitations}: The documentation of the project clearly mentions the limitations that exist (e.g., grammar checker with 10\% completion).
    
    \item \textbf{Accuracy Disclosure}: Performance measures and accuracy levels are provided truthfully.
    
    \item \textbf{Data Source Attribution}: Data from Wikipedia and other sources are properly credited.
\end{itemize}

\subsection{Intellectual Property and Licensing}
The initiative protects intellectual property but also encourages sharing:

\begin{itemize}
    \item \textbf{Open License}: This project has an open source license that allows academic and commercial use.
    
    \item \textbf{Data Compliance}: The training data used is all publicly available resources that have proper licensing (such as Wikipedia, which is available under CC BY-SA).
    
    \item \textbf{Attribution}: The citations are used to acknowledge previous work and research contributions.
\end{itemize}

\subsection{Bias and Fairness}
The project tackles issues of possible bias in natural language processing:

\begin{enumerate}
    \item \textbf{Corpus Diversity}: Even though the primary source is Wikipedia, the corpus represents various subject matters and writing patterns.
    
    \item \textbf{Regional Considerations}: The system is developed to support standard Bangla (Shudh Bangla) employed in Bangladesh and West Bengal, recognizing that some dialectal variations may not be fully supported.
    
    \item \textbf{Continuous Improvement}: The open-source model allows the community to identify and correct biases over time.
\end{enumerate}

\subsection{Professional Responsibility}
As software engineers, we recognize our responsibility to:

\begin{itemize}
    \item Supply software that performs as described and advertised
    \item Report bugs and issues clearly
    \item Sustain the project to fix security vulnerabilities
    \item Respond to user feedback and contributions
\end{itemize}

\section{Regulatory Compliance}
Though the regulations for language processing tools are few, the project considers:

\begin{itemize}
    \item \textbf{GDPR Compliance}: Lack of data processing removes GDPR concerns
    \item \textbf{Bangladesh ICT Act}: This project does not create, store, or transmit inappropriate content
    \item \textbf{Academic Integrity}: The purpose of the tool is to aid in writing, not to produce plagiarized content
\end{itemize}

\section{Summary}
In conclusion, engineering aspects of Byakaron include a holistic approach that combines technical excellence and social responsibility. The project:

\begin{enumerate}
    \item Encourages the preservation of the Bangla language
    \item Practices environmentally sustainable computing practices
    \item Upholds high ethical standards of data privacy and transparency
    \item Gives back to the open source community for sustainability
\end{enumerate}

Balancing all these elements is important to ensure that Byakaron is a responsible engineering solution that benefits the Bangla-speaking community while minimizing negative externalities.
